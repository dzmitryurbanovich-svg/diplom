\chapter*{ \large ЗАКЛЮЧЕНИЕ}
\addcontentsline{toc}{chapter}{ЗАКЛЮЧЕНИЕ}

В ходе работы были исследованы различные подходы к разработке сетевых приложений, существующие одноранговые сети, а также проблемы, связанные с разработкой peer-to-peer приложений, существующие в современной сети Интернет. 

Результатом работы стала библиотека для развёртывания peer-to-peer сети на языке программирования Rust, реализующая функционал для снятия ограничения "один пользователь --- одно устройство"\, и позволяющая реализовывать прикладные приложения на основе предложенной новой топологии. Библиотека разработана с целью упрощения разработки и открытия новых возможностей при проектировании peer-to-peer приложений.

В перспективе ставится задача расширения API библиотеки и развития предложенных в ней концепций и протоколов.

Peer-to-Peer сети являются интересной технологией. С развитием и ростом сети Интернет их использование стало менее удобным. Однако, P2P сети являются полезным базисом для создания различных приложений. Peer-to-peer сети и Blockchain являются основой Web3.0 --- будущим поколением сети Интернет --- и YAP2P может стать его частью. 