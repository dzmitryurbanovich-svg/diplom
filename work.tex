\chapter{Описание игры и целевая задача}

\section{Правила и особенности игры Каркассон}
Игра «Каркассон» (Carcassonne) является настольной стратегической игрой немецкого стиля (Eurogame), в которой игровой процесс строится на пошаговом выкладывании квадратов местности (тайлов) и размещении на них фишек подданных (миплов)~\cite{karna2012carcassonne}. 
Игровое пространство формируется динамически в процессе игры. Каждый новый тайл должен быть выложен по правилам топологической связности: грани выкладываемого квадрата должны логически продолжать объекты на уже лежащих на столе квадратах (дорога должна соединяться с дорогой, город с городом, поле с полем).

\section{Проблематика алгоритмического решения}
С точки зрения теории игр, Каркассон представляет собой стохастическую игру с полной информацией. Однако разработка оптимальной стратегии для искусственного агента сталкивается с рядом существенных трудностей:

\begin{itemize}
    \item \textbf{Размер пространства состояний.} В базовой версии игры участвуют 72 тайла, которые могут быть выложены в различных конфигурациях. Количество допустимых позиций возрастает экспоненциально с каждым ходом~\cite{heyden2009implementing}.
    \item \textbf{NP-трудность.} Задача поиска оптимального размещения тайлов математически эквивалентна обобщенной задаче подбора краев (Edge-Matching Puzzle), которая в общем случае является NP-полной.
    \item \textbf{Проблема горизонта планирования.} Из-за механизма подсчета очков (в частности, отложенного вознаграждения за владение полями в самом конце игры) классические жадные алгоритмы (Greedy algorithms) принимают локально оптимальные, но глобально убыточные решения~\cite{sutton2018reinforcement}.
\end{itemize}

\chapter{Математическая модель игры}

\section{Графовое представление игрового поля}
Для формализации задачи игровое поле моделируется как динамический планарный граф $G = (V, E)$, где:
\begin{itemize}
    \item $V$ --- множество вершин, представляющих собой отдельные функциональные сегменты тайлов (фрагменты города, участки дороги, поля, монастыри).
    \item $E$ --- множество ребер, обозначающих топологические связи между сегментами как внутри одного тайла, так и между смежными тайлами на игровом поле.
\end{itemize}

При добавлении нового тайла на поле граф $G$ модифицируется: добавляются новые вершины и строятся новые ребра, объединяющие смежные сегменты. Пример графового представления игрового поля представлен на рисунке~\ref{fig:graph_repr}.

\begin{figure}[h]
    \centering
    \includegraphics[width=0.8\textwidth]{page_4_img_4.png}
    \caption{Графовое представление игрового поля}
    \label{fig:graph_repr}
\end{figure}

\section{Отслеживание компонент связности}
Для эффективного управления связностью объектов (например, для проверки того, завершен ли город или дорога) в программной модели используется структура данных \textit{система непересекающихся множеств} (Disjoint-Set Union, DSU). DSU позволяет объединять независимые сегменты и находить корневой элемент компоненты за амортизированное почти линейное время $O(\alpha(n))$~\cite{knuth2000dancing}, что является критически важным для производительности игрового движка при расчете возможных ходов.

\chapter{Разработка гибридного алгоритма оптимизации}

\section{Архитектура системы}
Для решения задачи предлагается двухуровневая архитектура агента:
\begin{enumerate}
    \item \textbf{Логический уровень (Logic Layer):} Игровой движок, написанный на языке Python, который инкапсулирует в себе строгую математику, геометрию на сетке и правила игры. Он реализован в виде сервера Model Context Protocol (MCP)~\cite{anthropic2024mcp}, который детерминировано рассчитывает все возможные (допустимые с точки зрения правил) варианты хода в текущем состоянии.
    \item \textbf{Стратегический уровень (Strategy Layer):} Эвристическая функция оценки и выбора оптимального хода, реализованная на базе локальной Large Language Model (LLM) через Ollama. LLM используется для генерации высокоуровневых стратегических целей (например, «заблокировать город противника» или «занять перспективное поле»)~\cite{yao2024tree}.
\end{enumerate}

Связь между математическим движком и стратегическим ИИ осуществляется поверх стандарта JSON-RPC 2.0 с использованием спецификации Model Context Protocol. Концептуальная схема архитектуры решения представлена на рисунке~\ref{fig:arch_schema}.

\begin{figure}[h]
    \centering
    \includegraphics[width=0.8\textwidth]{page_6_img_0.png}
    \caption{Схема архитектуры решения (Logic Layer и Strategy Layer)}
    \label{fig:arch_schema}
\end{figure}

\section{Использование LLM в качестве эвристики}
Традиционные алгоритмы, такие как Monte Carlo Tree Search (MCTS)~\cite{ameneyro2020playing}, требуют симуляции миллионов случайных партий для оценки состояния. В предложенной модели языковая модель выступает в роли интеллектуальной эвристики, анализирующей текущую конфигурацию графа и сужающей дерево поиска на основе заложенных в нее паттернов стратегического планирования.
