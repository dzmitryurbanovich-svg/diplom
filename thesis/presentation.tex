\documentclass[10pt]{beamer}
\usepackage[english,russian]{babel}
\usepackage{fontspec}
\setmainfont{Times New Roman}
\setsansfont{Arial}
\setmonofont{Courier New}

\usetheme{Madrid}
\usecolortheme{whale}

\usepackage{graphicx}
\graphicspath{ {./images} }

\title[Математическая модель Каркассон]{Математическая модель и задача оптимизации игры Каркассон}
\subtitle{Глава 1: Описание игры и целевая задача}
\author[Урбанович Д. И.]{Дмитрий Игоревич Урбанович}
\institute[БГУ]{Белорусский государственный университет \\ Механико-математический факультет}
\date[Минск 2025]{Минск, 2025}

\begin{document}

\begin{frame}
    \titlepage
\end{frame}

\begin{frame}{Введение: Настольные игры как бенчмарк ИИ}
    \begin{itemize}
        \item \textbf{Эволюция ИИ:} От детерминированных игр (Шахматы, Го) к стохастическим средам.
        \item \textbf{Carcassonne (2000):} «Золотой стандарт» современных Eurogames.
        \item \textbf{Особенности для ИИ:}
            \begin{itemize}
                \item Динамическое игровое поле.
                \item Оптимизация ресурсов (миплы).
                \item Стохастический выбор (закрытая колода).
            \end{itemize}
    \end{itemize}
\end{frame}

\begin{frame}{Структура игрового процесса}
    \begin{columns}
        \begin{column}{0.6\textwidth}
            Каждый ход состоит из 3 фаз:
            \begin{enumerate}
                \item \textbf{Размещение тайла:} Соблюдение топологической связности.
                \item \textbf{Размещение мипла:} Захват объектов (дороги, города, монастыри, поля).
                \item \textbf{Подсчет и возврат:} Получение очков и освобождение ресурсов.
            \end{enumerate}
        \end{column}
        \begin{column}{0.4\textwidth}
            \includegraphics[width=\textwidth]{carcassonne_rules.jpeg}
        \end{column}
    \end{columns}
\end{frame}

\begin{frame}{Экономика ресурсов и стратегическая глубина}
    \begin{itemize}
        \item \textbf{Лимит миплов:} Всего 7 подданных на всю игру.
        \item \textbf{Проблема «замораживания»:} Риск потери мобильности на долгостроях.
        \item \textbf{Конкуренция:} Механики слияния и разделения объектов.
        \item \textbf{Мета-игра:} Блокировка соперника («мертвые города») и психологическое давление.
    \end{itemize}
\end{frame}

\begin{frame}{Система подсчета очков (Таксономия)}
    \begin{table}
        \begin{tabular}{|l|l|l|}
        \hline
        \textbf{Объект} & \textbf{Очки} & \textbf{Сложность ИИ} \\ \hline
        Дороги & 1/тайл & Низкая (Тактика) \\ \hline
        Города & 2/тайл + бонусы & Средняя (Геометрия) \\ \hline
        Монастыри & 9 (фикс) & Средняя (Окружение) \\ \hline
        Поля & 3 за кажд. город & Высокая (Стратегия) \\ \hline
        \end{tabular}
    \end{table}
\end{frame}

\begin{frame}{Комбинаторная сложность}
    \begin{itemize}
        \item \textbf{Пространство состояний:} $\sim 10^{40}$ (нижняя граница).
        \item \textbf{Дерево игры:} $\sim 10^{194}$.
        \item \textbf{Branching Factor:} До 500+ вариантов на один ход в середине партии.
        \item \textbf{Сравнение с шахматами:} Сложнее из-за динамического поля и стохастичности.
    \end{itemize}
\end{frame}

\begin{frame}{Проблематика алгоритмического решения}
    \begin{itemize}
        \item \textbf{Action Gap:} Разрыв между стратегическим планом ИИ и тактическим выбором координат.
        \item \textbf{Неопределенность:} Неизвестности будущего тайла требует вероятностного анализа.
        \item \textbf{Ограничения MCTS:} Стохастический шум мешает оценке долгосрочных полевых стратегий.
    \end{itemize}
\end{frame}

\begin{frame}{Сравнительный анализ подходов}
    \begin{enumerate}
        \item \textbf{Эвристики (Star2.5):} Быстро, но предсказуемо и без стратегии.
        \item \textbf{MCTS:} Мощный поиск, но «шумный» в Каркассоне.
        \item \textbf{Hybrid AI (LLM + Logic):}
            \begin{itemize}
                \item \textbf{Генерал (LLM):} Стратегическое планирование.
                \item \textbf{Солдат (Logic):} Тактическое исполнение.
            \end{itemize}
    \end{enumerate}
\end{frame}

\begin{frame}{Цели и задачи исследования}
    \textbf{Цель:} Разработка гибридной архитектуры ИИ (LLM + MCTS) для оптимизации стратегии.
    \vspace{0.3cm}
    \textbf{Задачи:}
    \begin{enumerate}
        \item Реализация MCP-сервера для игрового движка.
        \item Разработка агента «Генерал-Солдат».
        \item Исследование методик \textbf{Tree of Thoughts} и \textbf{Reflexion}.
        \item Сравнительный анализ эффективности в турнирах.
    \end{enumerate}
\end{frame}

\begin{frame}
    \begin{center}
        {\Huge Спасибо за внимание!} \\
        \vspace{1cm}
        Вопросы?
    \end{center}
\end{frame}

\end{document}
