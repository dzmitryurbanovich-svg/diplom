\documentclass[10pt, aspectratio=169]{beamer}
\usepackage[english,russian]{babel}
\usepackage{fontspec}
\setmainfont{Times New Roman}
\setsansfont{Arial}
\setmonofont{Courier New}

% Упрощенное оформление
\usetheme{default}
\usecolortheme{dove} 
\setbeamertemplate{navigation symbols}{}
\setbeamertemplate{footline}[frame number]

\usepackage{graphicx}
\usepackage{pgfplots}
\pgfplotsset{compat=1.18}
\graphicspath{ {./images} }

% Кастомный титульный лист в стиле БГУ
\setbeamertemplate{title page}{
    \begin{center}
        \fontsize{8pt}{10pt}\selectfont
        МИНИСТЕРСТВО ОБРАЗОВАНИЯ РЕСПУБЛИКИ БЕЛАРУСЬ \\
        БЕЛОРУССКИЙ ГОСУДАРСТВЕННЫЙ УНИВЕРСИТЕТ \\
        МЕХАНИКО-МАТЕМАТИЧЕСКИЙ ФАКУЛЬТЕТ \\
        Кафедра дифференциальных уравнений и системного анализа \\
        \vspace{1.0cm}
        \fontsize{12pt}{14pt}\selectfont
        \textbf{УРБАНОВИЧ Дмитрий Игоревич} \\
        \vspace{0.8cm}
        \fontsize{14pt}{16pt}\selectfont
        \textbf{МАТЕМАТИЧЕСКАЯ МОДЕЛЬ И ЗАДАЧА ОПТИМИЗАЦИИ ИГРЫ КАРКАССОН} \\
        \vspace{0.3cm}
        \fontsize{10pt}{12pt}\selectfont
        Дипломная работа \\
        \vspace{1.0cm}
        \hfill \fontsize{10pt}{12pt}\selectfont
        \begin{tabular}{l}
            Научный руководитель: \\
            ст. преподаватель И. И. Козлов
        \end{tabular}
        \vfill
        Минск, 2025
    \end{center}
}

\begin{document}

\begin{frame}[plain]
    \titlepage
\end{frame}

\begin{frame}{Текущее состояние разработки}
    На текущий момент реализованы следующие компоненты системы:
    \begin{itemize}
        \item \textbf{Игровой движок (Carcassonne C3):} Полная поддержка правил размещения тайлов, миплов и автоматического подсчета очков.
        \item \textbf{Интеллектуальные агенты:}
            \begin{itemize}
                \item \textbf{MCTS:} Стратегическое планирование с использованием дерева поиска.
                \item \textbf{Star2.5:} Оптимизированный эвристический поиск.
                \item \textbf{Hybrid LLM:} Архитектура «Генерал-Солдат», объединяющая LLM (Llama 3.3) и эвристический движок.
            \end{itemize}
        \item \textbf{Инфраструктура:}
            \begin{itemize}
                \item MCP-сервер для взаимодействия модели с состоянием игры.
                \item Система централизованной телеметрии и сбора данных для обучения.
            \end{itemize}
    \end{itemize}
\end{frame}

\begin{frame}{Анализ эффективности моделей: Успешность ходов}
    \begin{center}
    \begin{tikzpicture}
    \begin{axis}[
        ybar,
        ylabel={Процент успешных ходов (\%)},
        symbolic x coords={Greedy, Hybrid (Local), Hybrid (Cloud), LLM Alone},
        xtick=data,
        nodes near coords,
        nodes near coords align={vertical},
        ymin=0, ymax=110,
        width=10cm,
        height=6cm,
        bar width=25pt,
    ]
    \addplot coordinates {(Greedy,100) (Hybrid (Local),90) (Hybrid (Cloud),90) (LLM Alone,0)};
    \end{axis}
    \end{tikzpicture}
    \end{center}
    \footnotesize \textit{LLM Alone не справляется с жесткими правилами (Invalid JSON/Coordinates), тогда как Гибридная модель решает «Action Gap».}
\end{frame}

\begin{frame}{Результаты турнира (21 ход)}
    \begin{center}
    \begin{tikzpicture}
    \begin{axis}[
        ybar,
        ylabel={Набранные очки},
        symbolic x coords={GreedyPlayer, Cloud\_General (70B)},
        xtick=data,
        nodes near coords,
        ymin=0, ymax=30,
        width=8cm,
        height=6cm,
        bar width=35pt,
    ]
    \addplot[fill=blue!50] coordinates {(GreedyPlayer,22) (Cloud\_General (70B),20)};
    \end{axis}
    \end{tikzpicture}
    \end{center}
    \footnotesize \textit{Гибридный агент Cloud\_General показывает конкурентоспособный результат, уступая эвристическому GreedyPlayer всего на 2 очка.}
\end{frame}

\begin{frame}{Архитектура «Генерал-Солдат»}
    Решение проблемы «Action Gap» через разделение уровней абстракции:
    \begin{itemize}
        \item \textbf{Уровень Стратега (LLM):} Анализ ситуации на поле, выбор глобальной цели (блокировка, экспансия, накопление очков).
        \item \textbf{Уровень Тактика (Code/Heuristic):} Поиск конкретных координат на сетке, соответствующих приказу «Генерала».
    \end{itemize}
    \begin{center}
        \includegraphics[height=3cm]{architecture_schema.jpeg}
    \end{center}
\end{frame}

\begin{frame}{Задачи на финальном этапе}
    \begin{enumerate}
        \item \textbf{Fine-tuning промптов:} Улучшение точности стратегических советов через Tree of Thoughts.
        \item \textbf{Автоматизация обучения:} Использование накопленных логов телеметрии для In-Context Learning.
        \item \textbf{Аналитика:} Проведение масштабных турниров (100+ партий) для получения статистически значимых данных.
        \item \textbf{Завершение текста диссертации:} Описание результатов экспериментов в Главах 2 и 3.
    \end{enumerate}
\end{frame}

\begin{frame}{Программная платформа: Развёртка на Hugging Face Spaces}
    \begin{columns}
        \begin{column}{0.52\textwidth}
            \includegraphics[width=\textwidth]{hf_ui_screenshot.png}
        \end{column}
        \begin{column}{0.46\textwidth}
            \textbf{Реализованный функционал:}
            \begin{itemize}
                \item Визуализация поля из настоящих тайлов Каркассона (PIL-рендеринг).
                \item Выбор агента для каждого из двух игроков через dropdown.
                \item Счет и количество миплов в реальном времени.
                \item Лог разговора: агент сообщает своё решение (т.е. «LLM THINKING»).
                \item Раздел телеметрии для скачивания и анализа партий.
            \end{itemize}
            \vspace{0.3cm}
            \footnotesize \textit{Система задеплоена на HF Spaces и доступна через браузер без установки ПО.}
        \end{column}
    \end{columns}
\end{frame}

\begin{frame}
    \begin{center}
        {\Huge Спасибо за внимание!} \\
        \vspace{1cm}
    \end{center}
\end{frame}

\end{document}
