\chapter*{ \large ЗАКЛЮЧЕНИЕ}
\addcontentsline{toc}{chapter}{ЗАКЛЮЧЕНИЕ}

В ходе выполнения дипломной работы была решена задача создания автономного игрового агента для стохастической настольной игры с полной информацией Каркассон. 

Разработанный программный комплекс состоит из двух интегрированных слоев: детерминированного математического ядра (Logic Layer), реализованного в виде сервера Model Context Protocol, и эвристического модуля генерации стратегий (Strategy Layer), работающего на базе локальной языковой модели Ollama.

Полученные результаты демонстрируют принципиальную возможность использования современных больших языковых моделей (LLM) для решения NP-трудных задач комбинаторной оптимизации на динамических планарных графах. Архитектура взаимодействия посредством вызова инструментов (Tool Use / MCP) позволила компенсировать известную проблему галлюцинаций LLM и делегировать строгую проверку топологической связности и подсчет очков четкому алгоритмическому движку.

В качестве дальнейшего развития разработанного подхода планируется расширение эвристик агента за счет интеграции методов подкрепленного обучения (Reinforcement Learning) и алгоритмов поиска по дереву состояний (Monte Carlo Tree Search), а также адаптация обобщенного математического ядра для других настольных игр со схожей механикой размещения.