\chapter{Описание игры и целевая задача}

\section{Контекст и история развития настольных стратегий}
Развитие интеллектуальных систем в области настольных игр прошло долгий путь от простейших эвристик для шашек до сверхмощных нейронных сетей для игры в Го~\cite{silver2017alphazero}. Однако большинство классических исследований фокусировалось на играх с детерминированным процессом и полной информацией. Игра «Каркассон» (Carcassonne), созданная Клаусом-Юргеном Вреде в 2000 году~\cite{wrede2000carcassonne}, представляет собой принципиально другой класс задач.

«Каркассон» относится к так называемым играм «немецкого стиля» (Eurogames), где акцент смещен с прямого конфликта на оптимизацию ресурсов и долгосрочное планирование. Популярность игры и ее глубокие математические основы сделали ее «золотым стандартом» для тестирования современных алгоритмов искусственного интеллекта (ИИ), работающих в условиях неопределенности и стохастического выбора.

\section{Архитектура и правила игрового процесса}
Игровой мир «Каркассона» формируется динамически в реальном времени. В базовой версии игры участвуют два или более игроков, которые поочередно выстраивают ландшафт средневековой провинции, используя наборы квадратных тайлов.

\subsection{Циклическая структура хода}
Каждый игровой шаг формализован и строго разделен на три последовательные фазы:
\begin{enumerate}
    \item \textbf{Фаза размещения тайла.} Игрок вытягивает случайный тайл из закрытой колоды. Тайл должен быть пристыкован к существующему полю по правилам топологической консистентности. Стороны тайла (город, дорога, поле) должны логически продолжать объекты на соседних тайлах.
    \item \textbf{Фаза размещения подданного (мипла).} Это ключевая стратегическая фаза. Игрок может выставить одну из своих ограниченных фишек (миплов) на один из объектов только что размещенного тайла. При этом объект не должен быть занят ранее.
    \item \textbf{Фаза завершения и возврата.} Если действие привело к закрытию города, завершению дороги или окружению монастыря, происходит мгновенный подсчет очков. Использованные миплы возвращаются в запас игрока, что позволяет использовать их в последующих ходах.
\end{enumerate}

\subsection{Экономика ограниченных ресурсов}
В отличие от многих стратегий, где ресурсы можно накапливать неограниченно, в Каркассоне игрок обладает всего 7 миплами. Это создает жесткое ограничение: «замораживание» мипла на крупном, но труднозавершаемом объекте (например, огромном строящемся городе) лишает игрока мобильности. Стратегическая задача ИИ состоит в поиске баланса между инвестициями в крупные объекты и поддержанием «ликвидности» запаса подданных.

\section{Таксономия игровых стратегий и подсчета очков}
Для разработки интеллектуального агента необходимо учитывать сложную систему весовых коэффициентов, связанных с различными типами ландшафта.

\begin{itemize}
    \item \textbf{Дороги (Roads).} Самый простой вид объектов. Дают 1 очко за каждый тайл в составе завершенной дороги. Эффективны для быстрого возврата миплов.
    \item \textbf{Города (Cities).} Основной источник очков. Дает 2 очка за каждый тайл и 2 очка за каждый щит (эмблему) в составе города. Требуют сложного геометрического планирования для закрытия.
    \item \textbf{Монастыри (Cloisters).} Изолированные объекты. Завершаются, когда монастырь окружают 8 других тайлов. Дают фиксированные 9 очков.
    \item \textbf{Поля (Fields).} Наиболее сложная часть правил. Миплы на полях (крестьяне) не возвращаются до конца игры. Очки за них начисляются только в финальном финале (3 очка за каждый завершенный город, граничащий с полем). Это требует от алгоритма умения заглядывать за горизонт планирования на 50-70 ходов вперед.
\end{itemize}

\section{Проблематика разработки интеллектуальных агентов}
С точки зрения теории игр, Каркассон является стохастической игрой с неполной информацией (в части будущих тайлов колоды). Это порождает специфические трудности для автоматизации~\cite{heyden2009implementing}:

\subsection{Комбинаторный взрыв и branching factor}
Количество допустимых позиций для тайла возрастает нелинейно. Если в начале игры вариантов всего 4-12, то к 30-му ходу количество вакантных мест на «границе» игрового поля может достигать 60. Учитывая 4 варианта поворота и 1-8 вариантов размещения мипла, количество ветвей в дереве поиска на один ход может превышать 500. Это делает невозможным применение традиционного исчерпывающего поиска (Brute Force).

\subsection{Проблема «Действенного разрыва» (Action Gap)}
Одной из центральных проблем, решаемых в данной работе, является разрыв между стратегическим видением («хочу построить город») и тактическим исполнением (какой из 20 доступных поворотов выбрать). Классические поисковые алгоритмы вроде MCTS~\cite{browne2012mcts} хорошо находят локальные максимумы, но плохо «понимают» концепцию блокировки противника или подготовки плацдарма для будущего города.

\subsection{Баланс стохастики и детерминизма}
Использование случайной колоды вносит элемент удачи. Эффективный агент должен не просто искать лучший ход, а максимизировать математическое ожидание выгоды, учитывая вероятность выпадения нужного тайла в будущем. Это требует интеграции методов вероятностного анализа в структуру принятия решения.

\section{Сравнительный анализ современных подходов}
В современной практике разработки игрового ИИ выделяются три основных направления, каждое из которых имеет свои достоинства и недостатки:

\begin{enumerate}
    \item \textbf{Эвристические методы (Heuristic Search).} Основаны на жестко заданных правилах оценки позиции. Примером является алгоритм Star2.5. Они быстры, но предсказуемы и лишены гибкости.
    \item \textbf{Поиск по дереву Монте-Карло (MCTS).} Статистический метод, основанный на тысячах случайных симуляций партии (rollouts). MCTS доминирует в играх с глубоким деревом поиска, но в Каркассоне он часто упускает долгосрочные «полевые» стратегии из-за стохастического шума в симуляциях.
    \item \textbf{Гибридные модели (Hybrid AI).} Сочетание классических алгоритмов с Large Language Models (LLM). Именно этот подход исследуется в данной работе, где LLM выступает в роли стратега («Генерала»), а поисковый движок --- в роли тактического исполнителя («Солдата»).
\end{enumerate}

Данная глава закладывает теоретический фундамент для последующей разработки и экспериментального сравнения указанных методов в рамках единой программной среды.

\section{Сравнительный анализ с классическими игровыми дисциплинами}
Для более глубокого понимания сложности Каркассона как задачи для ИИ, необходимо провести корреляцию с другими настольными играми, уже ставшими вехами в истории развития вычислительного интеллекта.

\begin{itemize}
    \item \textbf{Carcassonne vs Chess.} В шахматах игровое поле фиксировано ($8 \times 8$). Основная сложность заключается в глубине перебора вариантов. В Каркассоне само поле является переменной величиной, что требует от алгоритма динамического перестроения графа связности на каждом шагу.
    \item \textbf{Carcassonne vs Go.} В Го количество состояний на доске $19 \times 19$ превосходит количество атомов во вселенной, однако информация на доске всегда полна. В Каркассоне присутствует «Скрытый фактор» (Blind deck) --- неопределенность следующего тайла, что сближает его с карточными играми.
    \item \textbf{Carcassonne vs Poker.} Подобно Покеру, в Каркассоне критически важен учет вероятностей и риск-менеджмент. Однако, в отличие от покера, геометрическая составляющая (топология поля) накладывает на игрока жесткие логические ограничения, которые нельзя «заблефовать».
\end{itemize}

\section{Психологические аспекты и «Мета-игра»}
Цифровая реализация агента должна учитывать не только математическую выгоду, но и особенности человеческого восприятия. В партиях высокого уровня часто применяется стратегия «Психологического давления»:
\begin{itemize}
    \item \textbf{Блокировка ключевых объектов.} Умышленное размещение тайла таким образом, чтобы противник никогда не смог достроить свой город (так называемые «мертвые города»). Для ИИ это требует понимания геометрии «невозможных стыковок».
    \item \textbf{Захват полей через экспансию.} Постепенное подведение своих крестьян к чужому крупному полю. Это форма скрытого противостояния, которую трудно выразить через мгновенную эвристику.
\end{itemize}

\section{Постановка цели и задачи исследования}
На основе проведенного анализа предметной области, целью данной магистерской диссертации является разработка и исследование гибридной архитектуры искусственного интеллекта для игры в Каркассон, сочетающей в себе точность традиционных методов поиска и стратегическую гибкость больших языковых моделей.

Для достижения поставленной цели необходимо решить следующие задачи:
\begin{enumerate}
    \item Спроектировать и реализовать программную платформу для моделирования игрового процесса с поддержкой Model Context Protocol (MCP).
    \item Разработать гибридного агента, использующего концепцию «Генерал-Солдат» для разделения стратегического планирования и тактического хода.
    \item Исследовать применимость методик Tree of Thoughts (ToT) и Reflexion для улучшения качества ходов LLM-агента без дообучения весов модели.
    \item Провести серию экспериментальных турниров для сравнения эффективности гибридной модели с классическими алгоритмами (MCTS, Greedy, Heuristic).
    \item Проанализировать собранные данные телеметрии и сформировать рекомендации по дальнейшей оптимизации стратегий ИИ в стохастических средах.
\end{enumerate}

Решение этих задач позволит не только создать сильного игрового агента, но и внесет вклад в понимание того, как символьные и нейросетевые подходы могут эффективно взаимодействовать при решении сложных оптимизационных задач.

\chapter{Математическая модель игры}

\section{Формализация игровых объектов и правил}
Для перевода правил настольной игры в программную логику необходимо формализовать основные компоненты системы.

\subsection{Тайл как информационный объект}
Центральным объектом системы является тайл --- квадрат местности, разделенный на функциональные зоны. Следуя подходу К. Хейден~\cite{heyden2009implementing}, каждый тайл описывается как структура, содержащая информацию о четырех своих сторонах (север, юг, запад, восток) и центральной области. 
Каждая сторона тайла имеет определенный тип (поле, город, дорога), что обеспечивает топологическую связность при стыковке: сторона $S_1$ тайла $T_1$ может быть соединена со стороной $S_2$ тайла $T_2$ только если их типы идентичны ($type(S_1) = type(S_2)$).

\subsection{Графовое представление игрового поля}
Для программной реализации игровое поле моделируется как динамический планарный граф $G = (V, E)$, где:
\begin{itemize}
    \item $V$ --- множество вершин, представляющих собой элементарные сегменты тайлов.
    \item $E$ --- множество ребер, соединяющих сегменты одного типа как внутри тайла, так и на границах смежных тайлов.
\end{itemize}
Граф $G$ является динамическим, так как на каждом ходу $k$ в него добавляется подграф $G_{tile}$ нового тайла, и строятся ребра $E_{match}$, связывающие его с существующей структурой поля.

При добавлении нового тайла на поле граф $G$ модифицируется: добавляются новые вершины и строятся новые ребра, объединяющие смежные сегменты. Пример графового представления игрового поля представлен на рисунке~\ref{fig:graph_repr}.

\begin{figure}[h]
    \centering
    \includegraphics[width=0.8\textwidth]{graph_repr.png}
    \caption{Графовое представление игрового поля}
    \label{fig:graph_repr}
\end{figure}

\section{Алгоритмическая реализация связности}
Одной из ключевых задач игрового движка является отслеживание завершенности объектов (городов, дорог) и распределение очков между игроками.

\subsection{Использование системы непересекающихся множеств (DSU)}
Для эффективного управления компонентами связности в графе $G$ применяется структура данных \textit{Disjoint-Set Union} (DSU) с оптимизациями по рангу и сжатию путей. 
Каждый сегмент тайла принадлежит определенному множеству $S \in \mathcal{S}$. Операция $Find(v)$ позволяет определить, к какому логическому объекту (например, конкретному городу) принадлежит вершина $v$. Операция $Union(u, v)$ выполняется при стыковке тайлов, объединяя сегменты в единый игровой объект.

Использование DSU позволяет решать следующие задачи за амортизированное время $O(\alpha(n))$~\cite{knuth2000dancing}:
\begin{itemize}
    \item \textbf{Проверка завершенности:} Объект считается завершенным, если у всех входящих в него сегментов нет свободных (не пристыкованных) сторон.
    \item \textbf{Определение владельца:} Подсчет количества миплов каждого игрока в рамках одной компоненты связности.
    \item \textbf{Мгновенный пересчет очков:} Обновление игрового состояния сразу после добавления ребра в граф.
\end{itemize}

\chapter{Разработка гибридного алгоритма оптимизации}

\section{Архитектурное разделение слоев}
Архитектура программного комплекса построена на принципе полного разделения ответственности между правилами игры и стратегическим интеллектом. Это разделение реализовано через два ключевых уровня:

\subsection{Логический уровень (Logic Layer)}
Игровой движок, написанный на языке Python, инкапсулирует в себе:
\begin{itemize}
    \item \textbf{Справочник тайлов:} Описание всех 72 плиток базового набора.
    \item \textbf{Геометрический валидатор:} Проверка допустимости координат и ориентации тайла.
    \item \textbf{Расчетный модуль:} Учет очков и мониторинг завершенности объектов через DSU.
\end{itemize}
Logic Layer функционирует как изолированный сервер, передающий информацию только по стандартизированным протоколам.

\subsection{Стратегический уровень (Strategy Layer)}
Верхнеуровневый агент, использующий локальную Large Language Model (LLM). Он не знает правил «в жестком коде», но получает описание текущей ситуации и доступные инструменты (tools) через интерфейс сервера.

\section{Реализация Model Context Protocol (MCP)}
Для взаимодействия слоев выбран протокол \textit{Model Context Protocol} (MCP). Это позволяет превратить игровой движок в «умную периферию» для ИИ.

\subsection{Компоненты MCP-сервера}
Разработанный сервер предоставляет следующие абстракции:
\begin{itemize}
    \item \textbf{Resources:} Текстовое описание текущего состояния поля и списка оставшихся тайлов в колоде.
    \item \textbf{Tools:} Функции \texttt{get\_legal\_moves} (возвращает список координат и поворотов), \texttt{place\_tile} (совершает ход) и \texttt{get\_score}.
    \item \textbf{Prompts:} Предустановленные шаблоны системных промптов, которые подготавливают LLM к роли профессионального игрока в Каркассон.
\end{itemize}
Взаимодействие осуществляется через JSON-RPC 2.0, что обеспечивает слабую связанность компонентов и позволяет в будущем легко заменить стратегический уровень (например, использовать другую модель LLM) без изменения кода игрового движка.

\begin{figure}[h]
    \centering
    \includegraphics[width=0.8\textwidth]{architecture_schema.jpeg}
    \caption{Схема архитектуры решения (Logic Layer и Strategy Layer)}
    \label{fig:arch_schema}
\end{figure}

\section{Интеграция LLM и эвристики}
В основе стратегического уровня лежит использование локальной языковой модели (Ollama), которая выступает в роли интеллектуального советника.

\subsection{Методология Tree of Thoughts (ToT)}
Для решения задачи планирования применяется фреймворк \textit{Tree of Thoughts}~\cite{yao2024tree}. В отличие от стандартного пошагового вывода, ToT позволяет модели:
\begin{itemize}
    \item Генерировать несколько вариантов хода («мыслей»).
    \item Оценивать каждый вариант на предмет долгосрочной выгоды (Self-Evaluation).
    \item Выбирать наиболее перспективную ветвь развития или выполнять бэктрекинг в случае обнаружения тупиковых стратегий.
\end{itemize}

\subsection{Вербальное подкрепление через Reflexion}
Для минимизации ошибок и улучшения качества ходов без дорогостоящего дообучения модели внедрен подход \textit{Reflexion}~\cite{shinn2023reflexion}. После каждого совершенного хода или в случае неудачного исхода партии, Logic Layer формирует текстовый отчет об ошибках, который передается модели. Модель выполняет «вербальную рефлексию», записывая выводы в оперативную память контекста, что повышает точность последующих решений.

\section{Гибридный поиск и оптимизация}
Хотя LLM способна к высокоуровневому планированию, она может допускать ошибки в точных расчетах. Для компенсации этого разработана гибридная модель:
\begin{itemize}
    \item \textbf{MCTS как фильтр:} Традиционный Monte Carlo Tree Search~\cite{ameneyro2020playing} выполняет быструю оценку терминальных состояний для отсечения заведомо проигрышных ходов.
    \item \textbf{LLM как селектор:} Языковая модель выбирает финальный ход из 3-5 лучших вариантов, предложенных MCTS, основываясь на стратегическом контексте, который трудно формализовать в чистой математике (например, психологическое давление на оппонента).
\end{itemize}
Такой симбиоз позволяет достичь баланса между вычислительной точностью и стратегической гибкостью.
