\chapter*{\large ВВЕДЕНИЕ}  
\addcontentsline{toc}{chapter}{ВВЕДЕНИЕ}

Объектом исследования данной дипломной работы является стохастическая игра с полной информацией. В качестве среды для тестирования алгоритмов выбрана настольная стратегическая игра немецкого стиля (Eurogame) Каркассон. Особенностями данной игры являются пошаговое выкладывание тайлов и четкие правила топологической связности (дорога к дороге, город к городу).

Актуальность исследования обусловлена следующими факторами: пространство состояний и возможных решений в данной игре исключает возможность полного перебора вариантов. Задача оптимального размещения тайлов является NP-трудной и эквивалентна обобщенной задаче Edge-Matching Puzzle. Кроме того, наличие отложенного вознаграждения (например, начисление очков за поля в конце игры) значительно снижает эффективность классических жадных алгоритмов.

Целью дипломной работы является разработка гибридного алгоритма оптимизации стратегии, объединяющего символьные вычисления (Model Context Protocol) и эвристический поиск на базе больших языковых моделей (LLM).

В ходе выполнения дипломной работы были поставлены следующие задачи:

\begin{itemize}
    \item Описание игрового поля как динамического планарного графа, а размещения тайлов как задачи удовлетворения ограничений (CSP).
    \item Разработка игрового движка и сервера Model Context Protocol (MCP) для детерминированного расчета допустимости ходов и правил связности.
    \item Реализация интеграции и алгоритмов с использованием локальной LLM (Ollama) для генерации высокоуровневых стратегических гипотез.
    \item Проведение турнира агентов и сравнение гибридного подхода с существующими жадными алгоритмами.
    \item Оценка эффективности применения LLM в качестве эвристической функции.
\end{itemize}