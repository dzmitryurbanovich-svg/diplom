\chapter*{\large РЕФЕРАТ}  
\addcontentsline{toc}{chapter}{РЕФЕРАТ}

В дипломной работе \_\_ страниц, \_\_ иллюстраций, \_\_ источников, \_\_ приложения.

СТОХАСТИЧЕСКАЯ ИГРА, ОПТИМИЗАЦИЯ СТРАТЕГИИ, РАЗМЕЩЕНИЕ ТАЙЛОВ, LLM, MCP, PYTHON.

Объектом исследования дипломной работы является стохастическая игра с полной информацией на примере игры Каркассон.

Целью дипломной работы является разработка гибридного алгоритма оптимизации стратегии, объединяющего символьные вычисления (MCP) и эвристический поиск на базе LLM.

Для достижения поставленной цели были использованы: язык программирования Python, Model Context Protocol (MCP), а также большая языковая модель (LLM) через Ollama.

В дипломной работе получены следующие результаты:

\begin{enumerate}
    \item Разработан игровой движок и MCP-сервер для вычисления допустимых ходов.
    \item Реализована интеграция с LLM для генерации стратегических гипотез и выбора приоритетов.
    \item Проведен анализ применимости LLM в качестве эвристической функции для задач комбинаторной оптимизации на графах.
\end{enumerate}

Дипломная работа является завершенной, поставленные задачи решены, присутствует возможность дальнейшего развития подхода на другие настольные игры.

Дипломная работа выполнена автором самостоятельно.


\newpage
\chapter*{\large РЭФЕРАТ}
\addcontentsline{toc}{chapter}{РЭФЕРАТ}

У дыпломнай працы \_\_ старонак, \_\_ малюнкаў, \_\_ крыніц, \_\_ дадаткаў.

СТАХАСТЫЧНАЯ ГУЛЬНЯ, АПТЫМІЗАЦЫЯ СТРАТЭГІІ, РАЗМЯШЧЭННЕ ТАЙЛАЎ, LLM, MCP, PYTHON.

Аб'ектам даследвання дыпломнай працы з'яўляецца стахастычная гульня з поўнай інфармацыяй на прыкладзе гульні Каркасон.

Мэтай дыпломнай працы з'яўляецца распрацоўка гібрыднага алгарытму аптымізацыі стратэгіі, які аб'ядноўвае сімвальныя вылічэнні (MCP) і эўрыстычны пошук на базе LLM.

Для дасягнення пастаўленай мэты выкарыстоўваліся: мова праграмавання Python, Model Context Protocol (MCP), а таксама вялікая моўная мадэль (LLM) праз Ollama.

У дыпломнай працы атрыманы наступныя вынiкi:

\begin{enumerate}
    \item Распрацаваны гульнявы рухавічок і MCP-сервер для вылічэння дапушчальных хадоў.
    \item Рэалізавана інтэграцыя з LLM для генерацыі стратэгічных гіпотэз і выбару прыярытэтаў.
    \item Праведзены аналіз прымяняльнасці LLM у якасці эўрыстычнай функцыі для задач камбінаторнай аптымізацыі на графах.
\end{enumerate}

Дыпломная праца з'яўляецца завершанай, пастаўленыя задачы вырашаны, прысутнічае магчымасць далейшага развіцця падыходу на іншыя настольныя гульні.

Дыпломная праца выканана аўтарам самастойна.


\newpage
\chapter*{\large ABSTRACT}
\addcontentsline{toc}{chapter}{ABSTRACT}

Thesis project is presented in the form of an explanatory note of \_\_ pages, \_\_ figures, \_\_ references, \_\_ applications.

STOCHASTIC GAME, STRATEGY OPTIMIZATION, TILE PLACEMENT, LLM, MCP, PYTHON.

The research object of this thesis project is a stochastic game with full information, using the game Carcassonne as an example.

The purpose of the thesis is to develop a hybrid strategy optimization algorithm combining symbolic computation (MCP) and LLM-based heuristic search.

To achieve the goal, the following were used: Python programming language, Model Context Protocol (MCP), as well as a large language model (LLM) via Ollama.

The main results of the thesis project are as follows:

\begin{enumerate}
    \item A game engine and MCP server were developed to compute valid moves.
    \item Integration with LLM was implemented to generate strategic hypotheses and select priorities.
    \item An analysis of the applicability of LLM as a heuristic function for combinatorial graph optimization problems was conducted.
\end{enumerate}

The thesis work is completed, the tasks set have been solved, there is the possibility of further development of the approach to other board games.

The thesis project was done solely by the author.

\newpage
